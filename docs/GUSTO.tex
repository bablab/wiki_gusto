% Options for packages loaded elsewhere
\PassOptionsToPackage{unicode}{hyperref}
\PassOptionsToPackage{hyphens}{url}
%
\documentclass[
]{book}
\usepackage{lmodern}
\usepackage{amssymb,amsmath}
\usepackage{ifxetex,ifluatex}
\ifnum 0\ifxetex 1\fi\ifluatex 1\fi=0 % if pdftex
  \usepackage[T1]{fontenc}
  \usepackage[utf8]{inputenc}
  \usepackage{textcomp} % provide euro and other symbols
\else % if luatex or xetex
  \usepackage{unicode-math}
  \defaultfontfeatures{Scale=MatchLowercase}
  \defaultfontfeatures[\rmfamily]{Ligatures=TeX,Scale=1}
\fi
% Use upquote if available, for straight quotes in verbatim environments
\IfFileExists{upquote.sty}{\usepackage{upquote}}{}
\IfFileExists{microtype.sty}{% use microtype if available
  \usepackage[]{microtype}
  \UseMicrotypeSet[protrusion]{basicmath} % disable protrusion for tt fonts
}{}
\makeatletter
\@ifundefined{KOMAClassName}{% if non-KOMA class
  \IfFileExists{parskip.sty}{%
    \usepackage{parskip}
  }{% else
    \setlength{\parindent}{0pt}
    \setlength{\parskip}{6pt plus 2pt minus 1pt}}
}{% if KOMA class
  \KOMAoptions{parskip=half}}
\makeatother
\usepackage{xcolor}
\IfFileExists{xurl.sty}{\usepackage{xurl}}{} % add URL line breaks if available
\IfFileExists{bookmark.sty}{\usepackage{bookmark}}{\usepackage{hyperref}}
\hypersetup{
  pdftitle={Stress \& Early Childhood Microbiome Development: Analysis of Data from the GUSTO Study},
  pdfauthor={By: Fran Querdasi, Bridget Callaghan},
  hidelinks,
  pdfcreator={LaTeX via pandoc}}
\urlstyle{same} % disable monospaced font for URLs
\usepackage{longtable,booktabs}
% Correct order of tables after \paragraph or \subparagraph
\usepackage{etoolbox}
\makeatletter
\patchcmd\longtable{\par}{\if@noskipsec\mbox{}\fi\par}{}{}
\makeatother
% Allow footnotes in longtable head/foot
\IfFileExists{footnotehyper.sty}{\usepackage{footnotehyper}}{\usepackage{footnote}}
\makesavenoteenv{longtable}
\usepackage{graphicx,grffile}
\makeatletter
\def\maxwidth{\ifdim\Gin@nat@width>\linewidth\linewidth\else\Gin@nat@width\fi}
\def\maxheight{\ifdim\Gin@nat@height>\textheight\textheight\else\Gin@nat@height\fi}
\makeatother
% Scale images if necessary, so that they will not overflow the page
% margins by default, and it is still possible to overwrite the defaults
% using explicit options in \includegraphics[width, height, ...]{}
\setkeys{Gin}{width=\maxwidth,height=\maxheight,keepaspectratio}
% Set default figure placement to htbp
\makeatletter
\def\fps@figure{htbp}
\makeatother
\setlength{\emergencystretch}{3em} % prevent overfull lines
\providecommand{\tightlist}{%
  \setlength{\itemsep}{0pt}\setlength{\parskip}{0pt}}
\setcounter{secnumdepth}{5}
\usepackage{booktabs}
\usepackage{amsthm}
\makeatletter
\def\thm@space@setup{%
  \thm@preskip=8pt plus 2pt minus 4pt
  \thm@postskip=\thm@preskip
}
\makeatother
\usepackage[]{natbib}
\bibliographystyle{apalike}

\title{Stress \& Early Childhood Microbiome Development: Analysis of Data from the GUSTO Study}
\author{By: Fran Querdasi, Bridget Callaghan}
\date{Updated on: 2021-01-02}

\begin{document}
\maketitle

{
\setcounter{tocdepth}{1}
\tableofcontents
}
\hypertarget{introduction}{%
\chapter{Introduction}\label{introduction}}

\begin{center}\rule{0.5\linewidth}{0.5pt}\end{center}

This wiki contains information on the BABLab's analysis of data from the GUSTO Study relating to Fran's 251 research project, titled ``Stress \& Early Childhood Microbiome Development''.

\hypertarget{information}{%
\chapter{Information}\label{information}}

\hypertarget{summary}{%
\section{Summary}\label{summary}}

This wiki includes information, notes and protocols on the ``Stress \& Early Childhood Microbiome Development'' analysis project.

\hypertarget{abstract}{%
\subsection{Abstract}\label{abstract}}

\hypertarget{aims}{%
\subsection{Aims}\label{aims}}

Taken from the \href{https://ucla.app.box.com/file/740587024475}{proposal} submitted by Bridget describing the project.

Aim 1a: Identify changes in the microbiome across the first 8 years of life.

Aim 1b: Determine how these changes vary as a function of maternal childhood adversity, maternal prenatal mental health, maternal postpartum stressful life events, and child stressful life events in a community sample.

Aim 1c: Determine whether children's gut microbiome composition prospectively predicts symptoms of psychopathology and gastrointestinal distress, as well as a change in symptoms across time.

\hypertarget{background}{%
\subsection{Background}\label{background}}

\textbf{GUSTO study background:}

\href{http://www.gusto.sg/}{GUSTO} (Growing Up in Singapore Toward Healthy Outcomes) is a large birth cohort study conducted in Singapore. The primary aim of GUSTO is to understand how conditions in pregnancy and early childhood influence the health and development of women and children. The study now includes data on the children up to 8 years of age. Bridget learned about this study through \href{https://douglas.research.mcgill.ca/michael-meaney}{Michael Meaney}, and the BABLab was granted access to some of the data to analyze.

\textbf{General BABLab GUSTO access background:}

Gusto is a dataset accessed through Michael Meaney's group. It is \emph{restricted access} so only BABLab staff and students with granted access may examine and analyze the data and write papers.

Bridget has an agreement with Dr.~Meaney's group that she will be senior author on any publications to come out of this dataset for the Form A questions, and a student can be first author. There are certain members of Dr.~Meaney's team (including Dr.~Meaney) who will need to be middle authors on the manuscript.

This dataset is also restricted use, which means that we can use it in the exact ways outlined in Form A. Any other use of the data will need to be preapproved through use of Form A.

\textbf{BABLab Stress \& Early Childhood Microbiome Development project background:}

Early life stress (ELS) in both parents and children is associated with a higher incidence of youth mental health disorders, cognitive dysfunction, and altered neurobiology (Green et al.~2010; McLaughlin et al.~2010; Schickedanz et al.~2018; Callaghan et al.~2016; Callaghan et al.~2019; Gee et al.~2013). Studies in animals have indicated that early life stress also influences the gastrointestinal microbiome (Bailey and Coe 1999; Pusceddu et al.~2015), which is itself associated with emotionality, cognition and the brain (Vuong et al.~2017). (from the \href{https://ucla.app.box.com/file/740587024475}{proposal} submitted by Bridget describing this project)

Previous research has documented that ELS is associated with altered composition of the gut microbiome in childhood (Callaghan et al., 2019), and that the period of maximal sensitivity of the gut microbiome to environmental influence extends until around the fourth year of life (Yaksunenko et al., 2012). Within this early life ``sensitive period'', in addition to adversity experienced by the child from birth to 4 years, prenatal adversity exposure in the form of maternal psychiatric symptoms during pregnancy has been associated with gut microbiome composition (Hu et al., 2019). Finally, composition of the maternal microbiome, which is associated with composition in their newborns, has been found to differ as a function of mothers' own ELA exposure (Hantsoo et al., 2019). However, studies have not investigated whether the timing of adversity occurrence within the large early life ``sensitive period'' (i.e., mothers' ELA vs.~prenatal adversity vs.~post-birth adversity) influences the magnitude of effect it exerts on the gut microbiome.

While gut microbiome composition has been associated with mental illness across the lifespan (O'Mahony et al., 2017), another open question concerns whether differences in gut microbiome composition can predict mental health outcomes in childhood. A recent cross-sectional study of 5-7-year-old children found that gut microbiome composition was associated with concurrent symptoms of internalizing and externalizing psychopathology (Flannery et al., 2020); however, longitudinal studies evaluating whether gut microbiome composition can predict future symptom course have not yet been conducted.

\hypertarget{collaborators}{%
\subsection{Collaborators}\label{collaborators}}

The BABLab has been communicating with the following GUSTO researchers about the data:

\begin{itemize}
\tightlist
\item
  Li Ting (general information and requesting data)
\item
  Neerja Karnani (microbiome data)
\end{itemize}

The following people/groups have requested acknowledgment in publications with these data:

\begin{itemize}
\tightlist
\item
  The GUSTO Nutrition Team (for scoring child feeding practices, processing microbiome data)
\end{itemize}

Research mentors for Fran's 251 project include:

\begin{itemize}
\tightlist
\item
  Bridget Callaghan
\item
  Andrew Fuligni
\end{itemize}

\begin{center}\rule{0.5\linewidth}{0.5pt}\end{center}

\hypertarget{significanceimpact}{%
\section{Significance/Impact}\label{significanceimpact}}

The scale and scope of data analyzed in this study (i.e., a longitudinal investigation of more than 400 mother-child dyads with microbiome data) is unprecedented; the scant developmental research on these topics has been cross-sectional and included less than 200 participants. This project will provide novel insights into how relationships between ELA exposure, the gut microbiome, and mental health symptomatology unfold across early childhood. This information will significantly increase precision in the field's understanding of how ELA influences gut microbiome composition and will potentially establish longitudinal links between gut microbiome composition and child mental health for the first time.

\begin{center}\rule{0.5\linewidth}{0.5pt}\end{center}

\hypertarget{measures}{%
\section{Measures}\label{measures}}

\hypertarget{stress}{%
\subsection{Stress}\label{stress}}

\textbf{Mothers' ELS:} When their children were 4 years old, mothers retrospectively reported on their own experiences of childhood maltreatment using the Childhood Trauma Questionnaire.

\textbf{Maternal Prenatal Psychopathology:} At 26 weeks' gestation, mothers reported their state (current) and trait (typical) anxiety levels using the State-Trait Anxiety Inventory, and their recent depressive symptoms using the Beck Depression Inventory. We will use a sum score across all measures to index maternal symptoms of psychopathology during pregnancy - i.e., child prenatal ELS.

\textbf{Children's Postnatal ELS:} When their children were 4.5 years of age, mothers reported on the stressful life events their children had experienced (e.g., death of a family member, hospital stay) up to that point in their life, as well as the date that each event occurred, using the Life Experiences Questionnaire. To compute a measure of ELA experienced by children concurrent with their provided age 4 microbiome sample, we excluded any events that occurred after the date that each child collected the stool sample from that timepoint.

\hypertarget{gut-microbiome}{%
\subsection{Gut Microbiome}\label{gut-microbiome}}

Stool samples were obtained from each child at the following timepoints: 1 Week, 3 Weeks, 3 Months, 6 Months, 1 Year, 2 Years, 4 Years, and 8 Years of age. Samples were then sequenced using 16s rRNA (a marker of bacterial DNA). From this sequencing information, we will compute two taxonomic metrics to assess gut microbiome composition: alpha diversity (number of different bacterial groups present within an individual), and beta diversity (amount of overlap among bacterial groups between individuals). We may also compute additional metrics.

\hypertarget{parenting}{%
\subsection{Parenting}\label{parenting}}

\hypertarget{diet-information}{%
\subsection{Diet Information}\label{diet-information}}

\hypertarget{covariates}{%
\subsection{Covariates}\label{covariates}}

Associated with gut microbiome composition in previous research:

\begin{itemize}
\tightlist
\item
  Mode of delivery (cesarean section vs.~vaginal birth)
\item
  Breastfeeding status
\item
  Ethnicity (\href{https://www.tandfonline.com/doi/full/10.1080/19490976.2020.1756150}{study})
\end{itemize}

\begin{center}\rule{0.5\linewidth}{0.5pt}\end{center}

\hypertarget{analyses}{%
\section{Analyses}\label{analyses}}

In aim 1, we will use 16s rRNA sequencing to determine the taxonomic composition of microbiota. First we will map how richness (number of observed operational taxonomic units; OTUs) changes across age, and as a function of maternal childhood adversity and prenatal stress using mixed effects modelling. We will also examine how diversity (differences in composition between individual -- using Unifrac phylogenetic distance metrics) differs as a function of within-individual age change, maternal childhood adversity and maternal prenatal stress using an unsupervised Principal Coordinates Analysis clustering approach. Finally, we will use Spearman rank correlation to determine which bacteria change monotonically with age, and maternal childhood and prenatal adversity. (from the \href{https://ucla.app.box.com/file/740587024475}{proposal} submitted by Bridget describing this project)

\begin{center}\rule{0.5\linewidth}{0.5pt}\end{center}

  \bibliography{book.bib,packages.bib,ref.bib}

\end{document}
